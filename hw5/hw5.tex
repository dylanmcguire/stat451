% STAT451 HW5

\documentclass{article}
\usepackage{anysize}
\usepackage{amsmath}
\usepackage{amssymb}
\usepackage{graphicx}

\marginsize{2cm}{2cm}{2cm}{2cm}

\title{STAT451 HW5}
\author{Russell Miller}
\date{\today}

\begin{document}

\maketitle

% 3.27, 3.29, 3.30, 3.36, 3.37, 3.39, 3.41, 3.51, 3.53, 3.55, and 3.56.         |

\paragraph{3.27 The time to failure in hours of an important piece of electronic 
equipment used in a manufactured DVD player has the density function}
\begin{eqnarray*}
f(x) = \left\{ \begin{array}{ll}
	\left(\frac{1}{2000}\right)e^{\frac{-x}{2000}}, & x\geq 0,\\
	0, & x<0.
	\end{array} \right.
\end{eqnarray*}
\begin{enumerate}
\item[\textbf{a.}] \textbf{Find $F(x)$.}
\begin{eqnarray*}
F(x) = \int f(x)dx = \left\{ \begin{array}{ll}
	0, & x<0\\
	-e^{\frac{-x}{2000}}, & x \geq 0
	\end{array} \right.
\end{eqnarray*}

\item[\textbf{b.}] \textbf{Determine the probability that the component (and 
thus the DVD player) lasts more than 1000 hours before the component needs to 
be replaced.}
\begin{eqnarray*}
P(X>1000) = \int_{1000}^\infty \frac{e^{\frac{-x}{2000}}}{2000}dx = .6065
\end{eqnarray*}

\item[\textbf{c.}] \textbf{Determine the probability that the component fails
before 2000 hours.}
\begin{eqnarray*}
P(X<2000) = \int_0^{2000} \frac{e^{\frac{-x}{2000}}}{2000}dx = .6321
\end{eqnarray*}
\end{enumerate}

\paragraph{3.29 An important factor in solid missile fuel is the particle size 
distribution. Significant problems occur if the particle sizes are too large. 
From production data in the past, it has been determined that the particle size 
(in micrometers) distribution is characterized by}
\begin{eqnarray*}
f(x) = \left\{ \begin{array}{ll}
	3x^{-4}, & x>1,\\
	0, & \mbox{elsewhere}
	\end{array} \right.
\end{eqnarray*}
\begin{enumerate}
\item[\textbf{a.}] \textbf{Verify that this is a valid density function.\\}
From my notes, a Probability Density Function is any function $f(x)$ that
satisfies:
\begin{itemize}
\item $f(x) \geq 0$ for all real x,
\item $\int_{-\infty}^\infty f(x)dx = 1$,
\item For event A\\
	$P(x \in A) = \int_{x \in A} f(x)dx$
\end{itemize}
The first bullet is satisfied, there are no negative values of $f(x)$.
The second bullet can be verified by
\begin{eqnarray*}
\int_1^\infty f(x)dx = \frac{-1}{\infty} - \frac{-1}{1^4} = 1
\end{eqnarray*}
Since we're not given any events or probabilities, this is sufficient.

\pagebreak
\item[\textbf{b.}] \textbf{Evaluate $F(x)$.}
\begin{eqnarray*}
F(x) = \left\{ \begin{array}{ll}
	0, & x < 1\\
	-x^{-3}, & x \geq 1
	\end{array} \right.
\end{eqnarray*}

\item[\textbf{c.}] \textbf{What is the probability that a random particle 
from the manufactured fuel exceeds 4 micrometers?}
\begin{eqnarray*}
P(X>4) = \int_{4}^\infty 3x^{-4}dx = \frac{1}{64}
\end{eqnarray*}
\end{enumerate}

\paragraph{3.30 Measurements of scientific systems are always subject to
variation, some more than others. There are many structures for measurement 
error and statisticians spend a great deal of time modeling these errors. 
Suppose the measurement error $X$ of a certain physical quantity is decided 
by the density function}
\begin{eqnarray*}
f(x) = \left\{ \begin{array}{ll}
	k(3-x^2), & -1 \leq x \leq 1,\\
	0, & \mbox{elsewhere.}
	\end{array} \right.
\end{eqnarray*}
\begin{enumerate}
\item[\textbf{a.}] \textbf{Determine $k$ that renders $f(x)$ a valid density
function.\\}
In order for $f(x)$ to be a valid density function, the following must hold.
\begin{eqnarray*}
\int_{-1}^1 k(3-x^2)dx & = & 1\\
\int_{-1}^1 (3k-kx^2)dx & = & 1\\
\left. 3kx - \frac{x^3}{3k} \right|_{-1}^{1} & = & 1\\
\left. k(3x - \frac{x^3}{3}) \right|_{-1}^{1} & = & 1\\
\frac{8k}{3} - \frac{-8k}{3} & = & 1\\
\frac{16k}{3} & = & 1\\
k & = & \frac{3}{16}
\end{eqnarray*}

And we'll double check by plugging it in
\begin{eqnarray*}
\int_{-1}^1 \left(\frac{3}{16}\right)(3-x^2)dx & = & 1\\
& &  \blacksquare
\end{eqnarray*}

\end{enumerate}
\end{document}
